\section{Conclusions}
\label{sec:conclusion_futurework}

We introduced a novel approach to constructing auto-tuning search spaces for GPU kernels using an optimized Constraint Satisfaction Problem (CSP) solver, addressing the specific challenges posed by the complexity of auto-tuning and large search spaces. 
Our contributions, available to the CSP-solving and auto-tuning community in the open-source \href{https://pypi.org/project/python-constraint2/}{python-constraint} and \href{https://pypi.org/project/kernel-tuner/}{Kernel Tuner} packages, substantially outperform state-of-the-art methods in search space construction performance, enabling the exploration of previously unattainable problem scales in auto-tuning and related domains.

Through rigorous evaluation, we demonstrated that our optimized CSP-based approach reduces construction time by several orders of magnitude, even for search spaces with billions of possible combinations. 
On average over the evaluated real-world applications, our optimized method is four orders of magnitude faster than brute force, three orders of magnitude faster than the unoptimized CSP solver, and one to two orders of magnitude faster than the state-of-the-art in search space construction. 
Our optimized search space construction method reduces the construction time of real-world applications to sub-second levels, eliminating it as a substantial factor in the overall tuning process overhead. %, confirming the efficiency of this new approach to search space construction. 
This breakthrough allows researchers and developers to more effectively harness the performance potential of modern GPUs and provides an efficient generic solver for similar problem domains.

\textbf{Availability}: The methods presented in this work are available as user-friendly software packages, enabling straightforward adoption by the auto-tuning community and related fields. 
They can be installed with \lstinline{pip install python-constraint2} and \lstinline{pip install kernel-tuner}. 
Both python-constraint and Kernel Tuner are open-source software welcoming contributions. 
For more information, visit the \href{https://github.com/KernelTuner/kernel_tuner}{Kernel Tuner}\footnote{\url{https://github.com/KernelTuner/kernel_tuner}} and \href{https://github.com/python-constraint/python-constraint}{python-constraint}\footnote{\url{https://github.com/python-constraint/python-constraint}} repositories. 

\ifdoubleblind
\else
\textbf{Acknowledgments}: The CORTEX project has received funding from the Dutch Research Council (NWO) in the framework of the NWA-ORC Call (file number NWA.1160.18.316).
\fi
